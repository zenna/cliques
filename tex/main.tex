%%
%% This is file `lexample.tex',
%% Sample file for siam macros for use with LaTeX 2e
%%
%% October 1, 1995
%%
%% Version 1.0
%%
%% You are not allowed to change this file.
%%
%% You are allowed to distribute this file under the condition that
%% it is distributed together with all of the files in the siam macro
%% distribution. These are:
%%
%%  siamltex.cls (main LaTeX macro file for SIAM)
%%  siamltex.sty (includes siamltex.cls for compatibility mode)
%%  siam10.clo   (size option for 10pt papers)
%%  subeqn.clo   (allows equation numbners with lettered subelements)
%%  siam.bst     (bibliographic style file for BibTeX)
%%  docultex.tex (documentation file)
%%  lexample.tex (this file)
%%
%% If you receive only some of these files from someone, complain!
%%
%% You are NOT ALLOWED to distribute this file alone. You are NOT
%% ALLOWED to take money for the distribution or use of either this
%% file or a changed version, except for a nominal charge for copying
%% etc.
%% \CharacterTable
%%  {Upper-case    \A\B\C\D\E\F\G\H\I\J\K\L\M\N\O\P\Q\R\S\T\U\V\W\X\Y\Z
%%   Lower-case    \a\b\c\d\e\f\g\h\i\j\k\l\m\n\o\p\q\r\s\t\u\v\w\x\y\z
%%   Digits        \0\1\2\3\4\5\6\7\8\9
%%   Exclamation   \!     Double quote  \"     Hash (number) \#
%%   Dollar        \$     Percent       \%     Ampersand     \&
%%   Acute accent  \'     Left paren    \(     Right paren   \)
%%   Asterisk      \*     Plus          \+     Comma         \,
%%   Minus         \-     Point         \.     Solidus       \/
%%   Colon         \:     Semicolon     \;     Less than     \<
%%   Equals        \=     Greater than  \>     Question mark \?
%%   Commercial at \@     Left bracket  \[     Backslash     \\
%%   Right bracket \]     Circumflex    \^     Underscore    \_
%%   Grave accent  \`     Left brace    \{     Vertical bar  \|
%%   Right brace   \}     Tilde         \~}


\documentclass[final]{siamltex}

% definitions used by included articles, reproduced here for
% educational benefit, and to minimize alterations needed to be made
% in developing this sample file.

\newcommand{\pe}{\psi}
\def\d{\delta}
\def\ds{\displaystyle}
\def\e{{\epsilon}}
\def\eb{\bar{\eta}}
\def\enorm#1{\|#1\|_2}
\def\Fp{F^\prime}
\def\fishpack{{FISHPACK}}
\def\fortran{{FORTRAN}}
\def\gmres{{GMRES}}
\def\gmresm{{\rm GMRES($m$)}}
\def\Kc{{\cal K}}
\def\norm#1{\|#1\|}
\def\wb{{\bar w}}
\def\zb{{\bar z}}

% some definitions of bold math italics to make typing easier.
% They are used in the corollary.

\def\bfE{\mbox{\boldmath$E$}}
\def\bfG{\mbox{\boldmath$G$}}


%%%%%%%%%%%%%%%%%%%%%%%%%%%%%%%%%%%%%%
% Actual stuff starts here


%% encoding
\usepackage[utf8]{inputenc}
\usepackage[english]{babel}



%% Math packages
\usepackage{amsmath}
\usepackage{amsfonts}
\usepackage{amssymb}
\usepackage{mathrsfs}

\usepackage{color}
\usepackage{hyperref}
\usepackage{graphicx}

\newcommand{\mycomment}[1]{{\color{blue} #1}}

\title{Landscapes}

% The thanks line in the title should be filled in if there is
% any support acknowledgement for the overall work to be included
% This \thanks is also used for the received by date info, but
% authors are not expected to provide this.

\author{Zenna Tavares,Michael Schaub }
\begin{document}

\maketitle

\section{Extended summary of some of the old documents}
I think three topics lines of research have emerged to pursue further:
\begin{enumerate}
 \item \textbf{Robustness in terms of the full landscape.} Pretty much along the lines of the initial project the objective is to find robustness measures / characteristics from the full landscape and then generate computable heuristics from this. Since we can only analyze small graphs we have to go for weighted graphs.
Some immediate objectives: create and fully analyze some weighted small graphs; clarify relation of robustness measures to movesets if possible; determine proxies for robustness measures;
Potential outcomes: introduce landscape picture for community detection; clarify faults in creation of landscapes (Good and Clauset); highlight moveset dependence and different ways to think about robustness; introduce computable proxies for robustness (in a certain context).
\item \textbf{The intermediate Louvain landscapes.} As Louvain like algorithms for optimization are now nearly everywhere some more theoretical work appears to be insightful... In particular the connection of the intermediate Louvain steps to some aspects of the landscape (related to disconnectivity graphs?) seems worth investigating. Also the notion of the Louvain moveset as jumps in the landscapes and the relation to other optimization methods that change the landscape “on the fly” might be worth looking into.
Some immediate objective: Find a way to measure the “distance” between intermediate Louvain landscapes; variation of information? Other matrix measures? clarify relation to disconnectivity graphs and optimization paths on landscapes if possible; comparison for some graphs with well defined hierarchical structure in comparison with some ill-defined detection problem?
\item \textbf{Theoretical work: a more profound picture of complexity and landscapes.} Can we unify notions of robustness by considering a probabilistic approach towards the landscapes (see Zenna notes); to what extend can we separate the complexity of the problem from the complexity of the optimization, complexity of a problem instance; data uncertainty vs other uncertainties; is this probabilistic view compatible with other robustness notions that focus on the shape of the landscapes, etc.
Some immediate steps: read more / get a good introduction to complexity and optimization (do you have one?); clarify probabilistic robustness notions on simple (computationally feasible) examples
\end{enumerate}

\section{Problem complexity and Characterising Algorithms}
We want to gain an understanding and measure(s) of the complexity of an optimisation problem.
The rough intuition behind these projects is that some aspect of the topology of the landscape and the dynamics it defines are informative.
The major problem is that a landscape is defined with respect to algorithm dependent variables; in other words if you modify the permissible transitions between candidate solutions, the resulting landscape topologies can vary greatly.
So how can we say anything about the landscape of a problem without reference to one particular algorithm; how can we differentiate the difficulty of a problem from the quality of an algorithm?

Perhaps we can't; it seems to be the case that if you try to distil problem specific complexities by considering all possible algorithms, you get trivial results and no insight.
For example you can always create a hypothetical algorithm which solves the problem instantly, and many more which consider all possible candidates, or perhaps never halt at all.
But perhaps you can't compute these hypothetical algorithms in practice, which leads to the central idea: although with respect to all algorithms the complexity of a problem may have no meaning, perhaps we can characterise 1) classes of algorithms that we are typically interested in 2) the relationship between different kinds of problem and different kinds of algorithm.

There are many ways to characterise an algorithm, for example the function it evaluates or its representation in a programming language.
We have three dimensions of consideration: deterministic/probabilistic, halting/non-halting, computable in polynomial time / unbounded.
Initially let's consider deterministic algorithms, but include both halting and non-halting algorithms with no bounds on the computational time, and denote the set of these with $DXX$.

For these purposes we can take a behaviourist perspective and consider an algorithm as an ordered set of objective function evaluations $E = f(c_1),.., f(c_n)$ where $f(c)$ is a mapping from candidate solution $c$ to its cost (or quality).  That is, we are describing an algorithm $a$ as sequence of candidate proposals (Q: is it the candidates or the quality of the candidates. In other words, should we consider two algorithms identical if $\forall i,j$ $f(c_i) = f(c_j)$?).  Given a (finite?) set of candidate solutions $C$, algorithm $a \in DXX$ can be thought of as a sequence which is either finite or ends in an infinite repetition (limit cycle) $(a_1,..,a_n, a_1,..)$.  $|DXX|$ is always finite, since the algorithms are deterministic.  If we think of $DXX$ as a directed graph, it is identical to the configuration spaces found in cellular automata (Maybe a good avenue for exploration).  

Typically however we are interested in a problem with respect to efficient algorithms, and are also not interested in those which do not halt.  So let us define a new class $DPH \subset DXX$ where $a \in DPH$ iff $|A|$ is finite and the sequence is computable in a time polynomial in $N$, where $N$ is the size of the problem instance. (TODO: make this more concrete)

The hypothesis then is that some topolgical propeties of $DPH$ properties are informative on the difficulty of an optimisation process.

Q: What class of algorithms does this apply to?

Q: Do we need / should we impose stronger constraints on P, e.g. the orders.

Q: How to even think about finding $P$ in principle and to approximation.

Q: Can we relax the determinsitic

\section{Stability Blobs}
Q: What does robustness mean and why do we care?

To what extent can we unify the numerous notions of robustness with respect to combinatorial optimisation and the community detection problem in particular.  In particular: robustness with respect to the graph, markov time amd the neighbourhood of a solution.

Idea: is that we can generalise the notion of a stability ball.

\subsection{background}
Using the notation of \ref{sotskov} we can describe a discrete optimisation problem as follows:
Let $E = \left\{e_i,e_2,..,e_n\right\}$ be a given set and $T_n=\left\{t_1,t_2,....,t_n\right\} m \ge 2$ be a set of subsets of $E$ (trajectories).
Each weighting can be considered as a vector $A=(a_1,..,a_n)$

Informally, a ball is defined in $R_{kk}$ space around $A$.
The stability ball is the ball for which every matrix $\tilde{A'}$ belonging to the ball

-- Need to finish this description

Q: How to compute/approximate this stability blob

Q: How to represent something like uniform distributions over infinite spaces, e.g. with respect to time?

Q: What are the properties of the stability blob we are interested in.

Q: Does this have any relation to above measures, i.e. robustness w.r.t optimisation

\end{document}